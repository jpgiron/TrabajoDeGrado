

\noindent La complejidad de los sistemas tanto en Hardware como en Software se 
ha venido  incrementando significativamente y por ende la probabilidad que los 
sistemas presenten fallas aumenta~\cite{Schneider2004}. Diversas \'areas de 
las ciencias de la computaci\'on y matem\'aticas han propuesto diferentes 
metodolog\'ias que permiten obtener sistemas Hardware/Software con un m\'inimo 
de errores; dichas metodolog\'ias estructuran la descripci\'on del sistema a 
trav\'es de m\'etodos formales o no formales, los cuales se pueden evaluar 
implementando desde una prueba de caja negra \footnote{Prueba de caja negra: Es 
una prueba funcional que consiste en estimular el sistema a probar con las 
entradas apropiadas que debe recibir la aplicaci\'on, y revisar si las salidas 
son las deseadas; lo anterior se lleva a cabo sin tener conocimiento de la 
estructura/funcionamiento interno del sistema.} hasta el uso de m\'etodos 
formales.

Garantizar que un sistema no tenga errores se ha convertido en un reto cada vez 
m\'as dif\'icil para la ingenier\'ia y las ciencias de las matem\'aticas. 
Existen sistemas en los cuales no se admiten errores, por ejemplo en los 
sistemas de aviaci\'on, sistemas de control de plantas nucleares, etc. 
~\cite{Bowen2000}. Sin embargo, a pesar que existen m\'etodos de 
verificaci\'on, validaci\'on y prueba de modelos, la complejidad de los sistemas 
se ha venido incrementando con gran rapidez y la forma de detectar errores con 
m\'etodos separados es insuficiente~\cite{Bowen2002}. 

Durante d\'ecadas se ha visto que los m\'etodos formales (\textit{formal 
methods}) y las pruebas  (\textit{Testing})  han sido rivales;  en efecto, 
cientos de cient\'ificos defienden que verificar los modelos por medio de 
estructuras de l\'ogica matem\'atica bien definidas est\'a por encima de hacer 
simples pruebas funcionales, como lo son las de caja negra.  No obstante, 
recientemente se ha visto que los m\'etodos formales y las pruebas son 
complementarios, pero lastimosamente aplicar estas metodolog\'ias para 
garantizar sistemas libres de errores est\'a muy lejos de ser realidad~\cite{Bowen2002, Gaudel1996, Hierons2009}. 

Uno de los m\'as famosos m\'etodos formales usados en la industria es el Model 
Checking~\cite{Mikovski2009}, el cual es una t\'ecnica de verificaci\'on 
autom\'atica que dado un modelo y una propiedad formal determina si dicho modelo 
la satisface, y en caso que no pueda hacerlo es capaz de informarle al 
desarrollador d\'onde est\'a el error para corregirlo~\cite{Arias2012, 
Mikovski2009}. 

Por otra parte, uno de los lenguajes m\'as conocidos a nivel mundial para 
realizar pruebas funcionales de caja negra es \textit{Testing and Test Control 
Notation Version 3}, TTCN-3,~\cite{Willcock2011,  ETSI}, el cual ha 
sido desarrollado y estandarizado por \textit{European Telecommunication 
Standards Institute}, ETSI. TTCN-3 provee diversas caracter\'isticas en cuanto 
al manejo de mensajes, niveles de abstracci\'on, m\'odulos para codificaci\'on 
y decodificaci\'on de los mensajes, entre otras~\cite{Willcock2011,ETSI}.

El esquema que se pretende llevar a cabo en este trabajo de grado es: 

\begin{itemize}
\item Mostrar la especificaci\'on y descripci\'on de un sistema a partir de 
m\'aquinas de estados finitas usando el lenguaje \textit{Specification and 
Description Language} (SDL) y su interacci\'on usando \textit{Message Sequence 
Chart}\footnote{M\'as informaci\'on de MSC referirse a la recomendaci\'on Z.120 
disponible en:  \url{https://www.itu.int/rec/T-REC-Z.120-201102-I/en}}(MSC).
\item Una vez obtenido el modelo del sistema, mostrar c\'omo a partir de 
pruebas funcionales de tipo caja negra se puede ejercitar el modelo en los 
niveles m\'as finos  de abstracci\'on. 
\item Para comprobar que el sistema sea confiable, se verificar\'a formalmente 
algunas de las propiedades del sistema, las cuales se consideren m\'as 
cr\'iticas para su funcionamiento.
\item Para ejecutar los anteriores pasos, se tomar\'a como caso de estudio el sistema de parqueaderos de autom\'oviles de la Pontificia Universidad Javeriana 
Cali; \'este caso ser\'a \'util para mostrar que el trabajo en conjunto de 
pruebas y verificaciones formales, posibilita implementar sistemas m\'as confiables que pueden 
ser implementados.

\end{itemize}

Para la descripci\'on y especificaci\'on del sistema se va a hacer uso de SDL y 
MSC, para las pruebas funcionales de caja negra se va a utilizar TTCN-3, y 
finalmente para las verificaciones formales se hace uso de IFx\footnote{M\'as 
Informaci\'on de la herramienta: \url{http://www-if.imag.fr}}, la cual es una 
herramienta desarrollada por VERIMAG\footnote{Para mas informaci\'on acerca de VERIMAG consultar la siguiente p\'agina: \url{http://www-verimag.imag.fr/?lang=en}} que por medio del acceso al \'arbol 
abstracto desde una especificaci\'on de SDL la traslada a un lenguaje intermedio 
llamado IF,  y a trav\'es de este \'ultimo se puede verificar usando un 
algoritmo de \textit{Model Checking}\footnote{Model checking es una t\'ecnica 
de verificaci\'on que, dado el modelo del sistema bajo estudio y la propiedad 
requerida, permite decidir autom\'aticamente si la propiedad es satisfecha o no 
~\cite{Clarke1996,Hames2009}}, que ya viene 
integrado en la herramienta. Un factor diferenciador que se le quiere dar al 
trabajo de grado es poder hacer uso de los tres lenguajes mencionados 
anteriormente con un \'unico software que se llama \textit{Real Time Developer 
Studio}\footnote{M\'as Informaci\'on del proveedor del 
software:\url{http://www.pragmadev.com}}, RTDS, software desarrollado por la 
compa\~n\'ia Francesa \textit{Pragmadev}.

