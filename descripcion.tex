

\section{Planteamiento del Problema}

La complejidad de los sistemas de Software/Hardware se ha venido incrementando 
significativamente debido a las necesidades de la industria 
~\cite{Clarke1996,Schneider2004,AlexeiSernaA}; debido a estas exigencias, la 
confiabilidad de dichos sistemas es un tema que est\'a siendo abordado desde 
diversos campos de las ciencias de la computaci\'on y las matem\'aticas. Es 
necesario garantizar que los modelos a implementar satisfagan propiedades de 
funcionamiento y sobre todo que sean seguros en situaciones donde la vida humana 
est\'e en juego~\cite{Gaudel1996,Bowen2000}, como por ejemplo: en los sistemas 
involucrados en el campo de la aviaci\'on, y otros medios de transporte, etc.

Actualmente, entre las m\'as conocidas t\'ecnicas que existen para la 
detecci\'on y depuraci\'on de errores, se encuentran las pruebas funcionales a 
sistemas y el uso de m\'etodos formales; sin embargo, a pesar de su existencia, 
es cada vez m\'as complejo garantizar sistemas sin errores haciendo uso de una 
sola t\'ecnica~\cite{Gaudel1996}. Por ejemplo, los m\'etodos formales se 
fundamentan en estructuras de l\'ogica matem\'atica bien definidas que no son 
tan sencillas de usar en diferentes etapas del desarrollo del sistema, debido a 
que exigen una alta demanda de tiempo y contar con personal altamente entrenado 
~\cite{AlexeiSernaA}; por otra parte, tampoco ser\'ia conveniente usar s\'olo 
pruebas funcionales a sistemas de seguridad cr\'itica pretendiendo garantizar 
el correcto funcionamiento al cien por ciento.

Por lo anterior, dado el incremento acelerado tanto en la complejidad como en la 
aplicaci\'on de los sistemas embebidos, es necesario proveer sistemas con niveles de 
confiabilidad suficiente para su correcto funcionamiento, haciendo uso combinado de dos t\'ecnicas en una sola metodolog\'ia
para la detecci\'on de errores en modelos a implementar, en este caso se trata 
de los ya mencionados M\'etodos formales y Pruebas funcionales.

\subsection{Formulaci\'on}

¿Es posible minimizar el n\'umero de errores en un sistema, usando en conjunto 
t\'ecnicas de m\'etodos formales y pruebas sobre diferentes etapas de desarrollo 
del sistema?

\subsection{Sistematizaci\'on}

\begin{enumerate}

\item ¿Cu\'al lenguaje semi-formal permite expresar un sistema a trav\'es de 
m\'aquinas de estados finitas?
 \item ¿Por qu\'e es necesario representar los sistemas a trav\'es de lenguajes 
formales?
\item ¿C\'omo diferenciar en qu\'e etapas del desarrollo de un sistema es 
conveniente aplicar m\'etodos formales o pruebas?
\item ¿Qu\'e tipo de t\'ecnicas son apropiadas para verificar si un sistema 
satisface unas condiciones dadas?
\item ¿C\'omo usar estructuras de l\'ogica matem\'atica para verificar 
modelos?
\end{enumerate}

\subsection{Posible t\'itulo del trabajo de grado}
Para definir el t\'itulo del trabajo de grado se han especificado las fases y 
las herramientas a usar en la verificaci\'on del correcto funcionamiento de un 
sistema distribuido, el cual empieza desde su descripci\'on en un lenguaje que 
no tenga ambig\"uedades y que se pueda manipular f\'acilmente para optimizar, 
probar y verificar a trav\'es de m\'etodos formales. Por lo anterior se 
defini\'o el siguiente t\'itulo:
\\

\textbf{\textit{``MODELADO, PRUEBA Y VERIFICACI\'oN DE SISTEMAS DISTRIBUIDOS 
USANDO RTDS e IFx''}}


\section{Objetivos}

\subsection{Objetivo general}

Mostrar el proceso de verificaci\'on del correcto funcionamiento de un modelo 
que representa un sistema distribuido, empleando un caso de estudio a partir de 
la uni\'on de dos t\'ecnicas: M\'etodos Formales y pruebas funcionales de 
caja negra, en diferentes etapas del desarrollo de sistemas usando las 
herramientas RTDS e IFx.

\subsection{Objetivos Espec\'ificos}
\begin{enumerate}
 \item Comprender el correcto funcionamiento de la herramienta Real Time 
Developer Studio.
\item Implementar el caso de estudio usando un lenguaje que no introduzca 
errores por ambig\"uedades en la descripci\'on, el cual posteriormente ser\'a 
verificado por medio de m\'etodos formales y pruebas.
\item Determinar en qu\'e etapas del desarrollo de sistemas es pertinente usar 
m\'etodos formales o pruebas.
\item Implementar una bater\'ia de pruebas incrementales que sea apropiada con 
el modelo del caso de estudio y compatible con el lenguaje en el cual ha sido 
expresado.
\item Entender el funcionamiento de IFx para verificar formalmente el modelo 
del caso de estudio.
\item Definir las propiedades cr\'iticas del sistema que se desea verificar.
\item Seleccionar los tipos de observadores para la especificaci\'on de la 
propiedad a verificar.
\item Seleccionar las propiedades a verificar.
\item Implementar los observadores.
\end{enumerate}

\section{Justificaci\'on}

Tanto la verificaci\'on formal como las pruebas funcionales son metodolog\'ias 
que ayudan a la detecci\'on y correcci\'on de errores de modelos que se van a 
implementar. A lo largo del tiempo encontramos en la literatura que \'estas dos 
t\'ecnicas han sido rivales; sin embargo, recientemente se puede apreciar 
que las ciencias computacionales y las matem\'aticas proponen que la mejor forma 
de verificar si un modelo satisface unas condiciones dadas, implica 
necesariamente usar estas dos t\'ecnicas en conjunto~\cite{Gaudel1996}. 

En el a\~no 2001 en la Universidad de Brunel, Reino Unido, se cre\'o una red 
llamada M\'etodos Formales y Pruebas, conocida por sus siglas en ingl\'es como 
FORTEST que significa Formal Methods and Testing. Esta red fue financiada por el 
Consejo de Investigaci\'on de Ciencias F\'isicas e Ingenier\'ia, por sus siglas 
en ingl\'es EPSRC\footnote{Sitio Web de 
EPSRC: \url{http://www.epsrc.ac.uk/Pages/default.aspx}}. El enfoque principal 
de 
FORTEST era explorar caminos que 
permitieran mostrar que los m\'etodos formales y el software de pruebas son 
complementarios, y as\'i facilitar los enfoques y t\'ecnicas 	que permitan 
producir sistemas de alta 
calidad\footnote{Para mas detalles:
\url{http://gow.epsrc.ac.uk/NGBOViewGrant.aspx?GrantRef=GR/R43150/01 
}}.

Por lo tanto, el presente trabajo de grado pretende mostrar a lo largo de su 
desarrollo que es posible dise\~nar sistemas confiables usando tanto las pruebas 
como los m\'etodos formales para la detecci\'on y correcci\'on de errores, 
bas\'andose en las investigaciones de la red FORTEST.

\section{Alcances y Limitaciones}

\begin{itemize}
 \item El proceso de modelado del caso de estudio se har\'a por medio 
de un lenguaje que no posea ambig\"uedades adem\'as de soportarlo la herramienta 
Real Time Developer Studio, en este caso ser\'a SDL.
\item El modelo del caso de estudio se limitar\'a a especificar de manera 
general los procesos involucrados; algunos de ellos se asumir\'an como 
subsistemas, los cuales proporcionando una entrada retornan una salida, pero no 
se modelar\'a su comportamiento dado que no es de inter\'es del presente trabajo 
de grado desarrollarlo. Ej.: modelar la base de datos de usuarios del 
parqueadero. 
\item Las pruebas realizadas al modelo se har\'an en las primeras fases del 
desarrollo del mismo sin usar componentes paralelos de pruebas (PTCs) usando SDL, MSC y TTCN3.
\item Las verificaciones formales no se har\'an en el nivel de sistema, sino 
sobre m\'odulos espec\'ificos.
\item Se har\'a uso de la herramienta SDL2IF para la transformaci\'on del 
modelo descrito en SDL a IF, con el fin de verificarlo formalmente.
\item Los procesos de modelado y pruebas se har\'an usando la herramienta Real 
Time Developer Studio, y la parte de verificaci\'on formal se har\'a por medio 
de la herramienta IFx.
\end{itemize}

